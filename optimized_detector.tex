\input{text/preamble}
\begin{document}

\def\labauthors{Шиков А.П.}
\def\labgroup{0420ДМР1Г}
\def\labnumber{1}
\def\labtheme{Исследование рабочих характеристик оптимального обнаружителя сложных радиолокационных сигналов.}
% \renewcommand{\vec}{\mathbf}
% \renewcommand{\phi}{\varphi}
% \renewcommand{\hat}{\widehat}

\input{text/titlepage}

\section{Введение}
\section{Практическая часть}
\subsection{Задание 1}
\textit{В начале отчета привести блок- схемы оптимального приемника
радиолокационного сигнала с использованием корреляторов и
согласованного фильтра, объяснить назначение их элементов. Привести
теоретические формулы для РХП в случае обнаружения известного
сигнала и сигнала со случайной фазой и амплитудой.}


Согласованный фильтр — это линейный оптимальный фильтр, построенный исходя из известных
спектральных характеристик полезного сигнала и шума. Согласованные фильтры предназначены
для выделения сигналов известной формы на фоне шумов. Под оптимальностью понимается
максимальное отношение сигнал/шум на выходе фильтра, и так как фильтр линейный форма
сигнала на выходе остается неизменной.

По определению детектор огибающей  должен осуществлять измерение огибающей входного сигнала,
т.е. формировать выходной сигнал вида $u_{\text{вых}}(t) = K_\text{дет}А(t)$.

Пороговое устройство фильтрует сигнал в зависимости от амплитуды выходного сигнала.

устройство синхронизации запускает генерацию сигнала и интегрирование, а по
окончании этого процесса подключает к выходу интегратора пороговое
устройство.

Для обнаружителя сигнала со случаной фазой нужно использовать два идентичных коррелятора
на которые в качестве опорных подается излучаемый сигнал и его квадратура. На выходах
корреляторов формируются действительная I и мнимая Q составляющие некоторого
аналитического сигнала.  



\begin{figure}[h!]
	\centering
	\includegraphics[width =0.8\linewidth]{imgs/scheme1.png}
	\caption{Блок-схема оптимального обнаружителя известного сигнала на фоне белого гауссова
    шума}
	\label{fig:scheme1}
\end{figure}

\begin{figure}[h!]
	\centering
	\includegraphics[width =0.8\linewidth]{imgs/scheme2.png}
	\caption{Блок-схема обнаружителя сигнала со случайной фазой и амплитудой на фоне белого
    гауссова шума с использованием двух корреляторов (устройство синхронизации на данной
    схеме отсутствует, т.к. интегрирование ведется в скользящем окне)
    }
	\label{fig:scheme2}
\end{figure}

\begin{figure}[h!]
	\centering
	\includegraphics[width =0.8\linewidth]{imgs/scheme3.png}
	\caption{Блок-схема обнаружителя сигнала со случайной фазой и амплитудой на фоне
    белого гауссова шума с использованием согласованного фильтра и детектора огибающей}
	\label{fig:scheme3}
\end{figure}

\subsection{Задание 2}
Было проведено сравнение спектра и корреляционной функции ЛЧМ-сигнала для окна с
плоской вершиной при различных значениях разности начальной и
конечной частот. Полученные графики приведена на рис. \ref{fig:spec1}-\ref{fig:spec4}

\begin{figure}[h!]
    \centering
    \begin{minipage}{0.49\linewidth}
        \includegraphics[width =0.9\linewidth]{imgs/spec1.png}
    \end{minipage}
    \begin{minipage}{0.49\linewidth}
        \includegraphics[width =0.9\linewidth]{imgs/corr1.png}
    \end{minipage}
	\caption{Разница частот 0.5 ($f_{s}=0, f_{e}=0.5$). Разница фаз = 0}
	\label{fig:spec1}
\end{figure}

\begin{figure}[h!]
    \centering
    \begin{minipage}{0.49\linewidth}
        \includegraphics[width =0.9\linewidth]{imgs/spec2.png}
    \end{minipage}
    \begin{minipage}{0.49\linewidth}
        \includegraphics[width =0.9\linewidth]{imgs/corr2.png}
    \end{minipage}
	\caption{Разница частот 0.3 ($f_{s}=0.1, f_{e}=0.4$). Разница фаз = 0}
	\label{fig:spec2}
\end{figure}

\begin{figure}[h!]
    \centering
    \begin{minipage}{0.49\linewidth}
        \includegraphics[width =0.9\linewidth]{imgs/spec3.png}
    \end{minipage}
    \begin{minipage}{0.49\linewidth}
        \includegraphics[width =0.9\linewidth]{imgs/corr3.png}
    \end{minipage}
	\caption{Разница частот 0.1 ($f_{s}=0.2, f_{e}=0.3$). Разница фаз = 0}
	\label{fig:spec3}
\end{figure}

\begin{figure}[h!]
    \centering
    \begin{minipage}{0.49\linewidth}
        \includegraphics[width =0.9\linewidth]{imgs/spec4.png}
    \end{minipage}
    \begin{minipage}{0.49\linewidth}
        \includegraphics[width =0.9\linewidth]{imgs/corr4.png}
    \end{minipage}
	\caption{Разница частот 0.01 ($f_{s}=0.25, f_{e}=0.26$). Разница фаз = 0}
	\label{fig:spec4}
\end{figure}

Из полученных результатов видно, что при уменьшении разницы между начальной
и конечной частотой ширина корреляционной функции увеличивается, а ширина спектра уменьшается.

Увеличение ширины корреляционной функции объясняется \dots

Уменьшенеие ширины спектра объясняется уменьшением количества частотынх компонент, использованных 
в сигнале. Таким образм, при устремлении разницы частот к нулю, вид спектра будет приближаться к $\delta$-функции.


\subsection{Задание 3}
Построить на одном графике зависимости вероятности правильного
обнаружения от вероятности ложной тревоги для трех значений
отношения сигнал/шум при известном сигнале. Сделать то же для сигнала
со случайной фазой и случайной фазой и амплитудой. 

\subsection{Задание 4}
На графиках указать несколько одинаковых значений порога. Объяснить
различия между графиками. Используя формулы для РХП, сравнить
экспериментальные результаты с теоретическими.
\section{Вывод}

\end{document}
